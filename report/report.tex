\documentclass{article}

\usepackage{listings}

\author{Guillaume Duclos-Cianci}
\date{\today}
\title{Benchmarking tool\\to compare various tracing solutions}

\begin{document}

\maketitle

\section{Objective}

Create a simple program that benchmarks the getuid system call to extract the overhead introduced by various tracers.

\section{Structure}

\subsection{Program}

The benchmark is performed in the following manner. First a timestamp is taken using a call to $clock\_gettime$. The value is saved in a timespec table. The system call $getuid$ is then performed. This process is repeted over and over such that two consecutive timstamps represent the time necessary to perform $clock\_gettime$ plus $getuid$ plus the loop to next round.

Here is a snippet of the code performing the benchmark.
\begin{lstlisting}
    for(ts_iter, i; i < BURST_SIZE; ts_iter++, i++){
        clock_gettime(CLOCK_MONOTONIC_RAW, ts_iter);
        getuid();
    }   
    clock_gettime(CLOCK_MONOTONIC_RAW, ts_iter);
\end{lstlisting}

The program arguments are the sample size ($-s$), the number of threads ($-t$) and the number of cores to benchmark ($-n$).

\subsection{scripts}

Tracers:
\begin{enumerate}
	\item Lttng
	\item perf
	\item ftrace
	\item system tap
\end{enumerate}

\subsection{Lttng}

The parameters explored for Lttng are the number of subbuffers, \textit{num-subbuf}, and their size, \textit{subbuf-size}. We also compare standard output writtent to disk and snapshot mode.

\subsection{System tap}

Various probe contents were tested. In all cases, the scaling was prohibitive.

\end{document}
